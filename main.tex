\documentclass[10pt,a4paper]{article}
\usepackage{amsmath, amssymb}
\usepackage{amsthm}
\usepackage[hyphens]{url}
\usepackage[margin=2cm]{geometry}
\usepackage[english]{babel}

\newtheorem{theorem}{Theorem}[section]
\newtheorem{lemma}[theorem]{Lemma}
\newtheorem{premise}[theorem]{Premise}
\newtheorem{case}[theorem]{Case}

\begin{document}

\title{Acceptable Quantities of Fruits For Chinese Gifts}
\author{H. Chen}
\date{December, 2024}
\maketitle

\section{Introduction}
Gifting oranges is a common practice in Chinese culture. In this article,we investigate the ideal number of oranges to gift.

\section{Proof}
\begin{premise}
    In Chinese, Fruits and Numbers have additional meanings drawn from how similar their pronunciations are to other words. For example, the word for "apple" sounds like "peace" and "safety".
\end{premise}

\begin{premise}
    In Chinese Culture, numbers have special meanings. For example, the number 8 is considered lucky whilst 4 is considered unlucky.
\end{premise}

\begin{premise}
    The word "orange" is similar to the Chinese word for "success" or "wealth". Therefore, by handing someone orange, you are wishing them success.
\end{premise}

\begin{premise}
    Other fruits that are considered "auspicious" include apples (for safety), mandarins (good luck) and lychee (for profit).
\end{premise}

\begin{case}
    For the values of 3, 6, 8 and 9.

    \begin{itemize}
        \item 3: Represents Stability and Growth. 
        \item 6: Symbolizes smoothness and success.
        \item 8: The king of lucky Chinese numbers.
        \item 9: Means longevity and eternity.
    \end{itemize}

\end{case}

\begin{case}
    For the values of 4 and 7.

    \begin{itemize}
        \item 4: Represents death.
        \item 7: Represents anger.
    \end{itemize}
\end{case}

\section{Conclusion}
The ideal number of oranges to gift is 3, 6, 8 or 9.

Other fruits, if gifted in the given quantities, may also be good gifts (especially to enhance the different aspects of the recipient's life - other than wealth or success).
\end{document}